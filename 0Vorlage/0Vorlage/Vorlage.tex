\documentclass[
    toc=bibliographynumbered,
    DIV=13,
    12pt,
    twoside=false
]{article}


\setlength{\parindent}{0pt}
\usepackage[backend=biber, style=phys, biblabel=brackets, block=ragged, language=autobib, autolang=other]{biblatex}
\usepackage{newcomputermodern}

\usepackage{paralist}
\usepackage{tcolorbox}
\usepackage{tabularray}
\usepackage{caption}
\usepackage{titlesec}    % Für das Formatieren von Überschriften
\usepackage{lipsum}      % Zum Generieren von Blindtext
\usepackage{subfiles}
\usepackage{xspace}
\usepackage{boxedminipage2e}
\usepackage{fancyhdr}
\usepackage[ngerman]{babel}
\usepackage[pdfborderstyle={/S/U/W 1}]{hyperref}
\usepackage{amsmath}
\usepackage{siunitx}
\usepackage{graphicx}
\usepackage{float}
\usepackage{microtype}
\usepackage[figure]{hypcap}
\usepackage{tikz}
\usepackage{icomma}
\usepackage{csquotes}
\usepackage{booktabs}
\usepackage{ulem}
\usepackage{pdfpages}
\usepackage{xcolor}
\usepackage{upgreek}
\usepackage{tocloft}
\usepackage{hyperref}
\usepackage[a4paper,
left=2.5cm, right=2.5cm,
top=2.5cm, bottom=2cm, headheight = 2cm]{geometry}
\addbibresource{4.referenzen.bib}

\pagestyle{empty}

\newcommand{\Ueberschrift}{Ich bin ein Murmeltier}
\newcommand{\Datum}{31.01.2025}%
\newcommand{\kb}{\text{\begin{math} k_{\mathrm{B}} \end{math}}}
\renewcommand{\sin}[1]{\text{sin\begin{math}\left(#1\right)\end{math}}}
\renewcommand{\cos}[1]{\text{cos\begin{math}\left(#1\right)\end{math}}}
\newcommand{\nameee}{Party People}
\newcommand{\group}{Gruppe: Penis}
\renewcommand{\footrulewidth}{0.4pt}
\renewcommand{\headrulewidth}{0.4pt}

\newcommand{\var}[2]{\text{\begin{math} #1_{\mathrm{#2}}
\end{math} }}

\DeclareDocumentCommand{\bz}{smE{_}{}oO{,}E{^}{}oO{,}}{%  \bz* T_c[max]^2
\ensuremath%
{%
 \IfBooleanTF{#1}{\mathrm{#2}}{#2}%
 \IfNoValueF{#3}{%
  \sb%
  {%
   \mathrm{%
    #3\IfNoValueF{#4}{#5#4} %
   }%
  }%
 }%
 \IfNoValueF{#6}{%
  \sp%
  {%
   \mathrm{%
    #6\IfNoValueF{#7}{#8#7}%
   }%
  }%
 }%
}%
}%


\captionsetup{
    font={small}, % Schriftgröße (z. B. small, footnotesize, large)
    labelfont={color=black,bf}, % Farbe und Stil des Labels (z. B. "Abbildung 1")
    textfont={color=gray}, % Farbe des Textes
}



% Definiere die Farbe
\newcommand{\Farbe}{red!40!gray}


\renewcommand{\cftsecfont}{\color{\Farbe}\bfseries} % Kapitelüberschriften fett und rot
\renewcommand{\cftsecpagefont}{\color{black}}

%  \setlength{\cftbeforesecskip}{2cm}
% Setze Schriftart, Größe und Farbe für die Überschriften und behalte die Nummerierungen bei
\titleformat{\section}[hang]
  {\LARGE\bfseries\color{\Farbe}} % Format für Titel
  {\begin{tcolorbox}[colframe=red!40!gray, colback=red!40!gray, sharp corners, boxrule=0.8mm, arc=4mm, width=1cm, height=1cm, halign=center, valign=center] \textcolor{white}{\huge\thesection} \end{tcolorbox}} % Nummerierung in Box
  {1em} {}  % Abstand und Titel
%   \titlespacing*{\section}{0pt}{2ex plus 1ex minus .2ex}{3ex plus .2ex}

\titleformat{\subsection}[hang]
  {\Large\bfseries\color{\Farbe}} % Format für Titel
  {\begin{tcolorbox}[colframe=red!40!gray, colback=red!40!gray, sharp corners, boxrule=0.8mm, arc=4mm, width=1.4cm, height=0.8cm, halign=center, valign=center] \textcolor{white}{\Large\thesubsection} \end{tcolorbox}} % Nummerierung in Box
  {1em} {}  % Abstand und Titel

\titleformat{\subsubsection}[hang]
  {\large\bfseries\color{\Farbe}} % Format für Titel
  {\begin{tcolorbox}[colframe=red!40!gray, colback=red!40!gray, coltitle=white, sharp corners, boxrule=0.8mm, arc=4mm, width=1.7cm, height=0.6cm, halign=center, valign=center]\textcolor{white} {\large\thesubsubsection} \end{tcolorbox}} % Nummerierung in Box
  {1em} {}  % Abstand und Titel


\begin{document}
\begin{center}
    \vspace*{1cm}
    \begin{boxedminipage}[t][3.5cm][c]{\textwidth}
        \begin{center}
            \begin{Huge}
                \textbf{\Ueberschrift}
            \end{Huge}
                \begin{Large}
                    \\Anfängerpraktikum\\
                \end{Large}
                    \Datum
        \end{center}
    \end{boxedminipage}

    \vspace{2cm}

    \begin{boxedminipage}[t][10cm][c]{10cm}
        \includegraphics[width=\linewidth]{./Bilder/Aufbau/sie.JPG}	  
    \end{boxedminipage}

    \vspace{2cm}

    \Large{\nameee\\}
    \large{\group}
    \normalsize

    \vspace{2cm}

    \color{black}  
      \textbf{„Mit der Abgabe dieses Protokolls wird bestätigt, dass es kein Plagiat ist. Falls es dennoch eindeutig
      als Plagiat erkannt werden sollte, ist bekannt, dass das einen Punktabzug von 20 Punkten zur Folge,
      ohne Möglichkeit der Nachbearbeitung, hat. Diese Bewertung wird ausnahmslos zur Gesamtnote im
      Anfängerpraktikum beitragen.“}    
\end{center}

\pagestyle{fancy}
\fancyhead[L]{\Ueberschrift}
\fancyhead[R]{\Datum}
\renewcommand{\headrulewidth}{0.4pt}
\fancyfoot[L]{\nameee}
\renewcommand{\footrulewidth}{0.4pt}

\tableofcontents
\newpage

\section{Theorie}

% \begin{figure}[H]
%     \begin{center}
%     \includegraphics[width=\linewidth]{./Bilder/Auswertung/A1k.png}\\
%     \end{center}
%     \caption{Bestimmung der PTC-Propotionalitätskonstante $α$}
%     \vspace{0.1cm}
%     \begin{center}
%     Quelle: eigene Auswertung
%     \end{center}
%     \label{fig:A1}
% \end{figure}
\section{Aufbau und Durchführung}

% \begin{minipage}{\textwidth} 
%     \begin{tblr}{
%         hlines={gray,1pt},
%         colspec={XXXXXX},
%         cell{odd}{2-Z}={white!80!black},
%         cell{even}{2-Z}={white!90!black},
%         cell{odd}{1}={red!40!white},
%         cell{even}{1}={red!40!white},
%         row{1}={red!50!gray,fg=white,font=\bfseries},
%     }


%     \end{tblr}
%     \captionof{table}{NTC-Widerstandswerte}
%     \label{tab: tab2}
% \end{minipage}
% \vspace{0.3cm}

\section{Auswertung}


\section{Anhang}

\listoffigures
\listoftables

\printbibliography[title={\textcolor{black}{Quellenverzeichnis}}]






%\includepdf[pages=-]{W4.pdf}
\end{document}